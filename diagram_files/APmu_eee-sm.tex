% \author       Moritz Kiehn <kiehn@physi.uni-heidelberg.de>
% \copyright    Copyright (c) 2014 Moritz Kiehn
% \license      GNU Public License Version 2

\begin{fmffile}{APmu_eee-sm}
\begin{fmfgraph*}(150,80)
    \fmfstraight
    \fmfleft{i1}
    \fmfright{o1,o2,o3}

    % layout skeleton
    \fmf{phantom,tension=1.5}{l1,i1}
    \fmf{phantom,tension=2}{l1,center,l2}
    \fmf{phantom,tension=1}{l2,o2}
    \fmf{phantom,tension=0.4,left,tag=1}{l1,l2}
    \fmf{phantom,tension=0.4,left,tag=2}{l2,l1}
    \fmffreeze
    \fmf{phantom}{center,g1,o3}
    \fmffreeze
    \fmfipath{p[]}
    \fmfiset{p1}{vpath1(__l1,__l2)}
    \fmfiset{p2}{vpath2(__l2,__l1)}

    % incoming / outgoing lines
    \fmf{fermion,tension=1,label=\APmuon,label.side=left}{l1,i1}
    \fmf{fermion,tension=1,label=\APelectron,label.side=right,label.dist=0.001}{o3,g1}
    \fmf{fermion,tension=1,label=\Pelectron,label.side=right}{g1,o2}
    \fmfi{fermion,label=\APelectron,label.side=left}{vloc(__o1) -- point 1/6length(p2) of p2}
    \fmfi{photon,label=\Pphoton/\PZ}{point 5/6length(p1) of p1 -- vloc(__g1)}

    % loop content
    \fmfi{boson,label=\PWplus}{subpath (0,length(p1)) of p1}
    \fmfi{boson}{subpath (0,1/6length(p2)) of p2}
    \fmfi{plain,label=\HepProcess{\APnum\leftrightarrow\APnue}}{subpath (1/6length(p2),length(p2)) of p2}
    \fmfiv{d.shape=cross,d.size=5thick}{point 1/2length(p2) of p2}
\end{fmfgraph*}
\end{fmffile}

